% !TeX root = ../full_report.tex

\chapter{Evaluation}
\section{Datasets}\label{sec:datasets}
The proposed approaches as well as several base-lines are evaluated
on two datasets: the CLICIDE and GaRoFou datasets. These datasets are
described in detail by Portaz et al~\cite{portaz_construction_nodate}.
Both datasets are typical of instance search datasets
in museums or touristic sites: the objects represented by their images
are paintings for one and glass cabinets containing sculptures and artifacts
for the other dataset. Both datasets contain a small number of images
per instance and a small number of images in total.
Table~\ref{tab:datasets} lists the content and statistics of the datasets
in detail.

\subsection{CLICIDE}
The CLICIDE dataset contains photographs taken in the Grenoble Museum of Art.
The objects represented in the photographs are paintings and still images,
exclusively. The full dataset contains 207 images with no meaningful content,
mostly showing walls. These images are labeled as such and have been filtered
from the dataset in our evaluation. This explains the difference between
total number of images and number of images in Table~\ref{tab:datasets}.
Furthermore, there are 12 query images which contain objects not
present in the reference images. These have been filtered out as well.

The CLICIDE dataset is characteristic because the different images
for each instance usually consist of at least one global view of the
painting and multiple other images representing sub-regions of the
same painting.

\subsection{GaRoFou}
The GaRoFou dataset contains high-resolution photographs taken in the
Museun of Fourvière in Lyon. The objects represented in the photographs
are a cultural heritage collection: sculptures, steles and small artifacts
grouped together in glass cabinets.

The full GaRoFou dataset is larger than the part used here: along with
the dataset of images, it contains a dataset of videos as well as images
extracted from these videos. The video collection has not been used as
part of our research.

\begin{table}
\begin{tabular}{|*{3}{l|}}
\hline & \emph{CLICIDE} & \emph{GaRoFou}\\
\hline \emph{Type of objects} & Paintings & Glass cabinets
containing various objects\\
\hline \emph{\#Instances} & 464 & 311\\
\hline \emph{\#Reference images} & 3245 & 1068\\
\hline \emph{\#Query images (\#instances)} & 165 (134) & 184 (166)\\
\hline \emph{Median \#images per instance} & 6 & 2\\
\hline \emph{Min \#images per instance} & 1 & 1\\
\hline \emph{Max \#images per instance} & 22 & 45\\
\hline \emph{Total reference images (\#instances)} & 3452 (473) & 1068 (311)\\
\hline \emph{Total query images (\#instances)} & 177 (143) & 184 (166)\\
\hline
\end{tabular}
\caption{Details of the datasets used for evaluation\label{tab:datasets}}
\end{table}

\section{Metrics}
There are several commonly used metrics in classification and information
retrieval. The following sections define these metrics and their use
in our research.

\subsection{Precision}
\begin{equation}\label{eq:preccls}
P = \frac{TP}{PRED}
\end{equation}

\begin{equation}\label{eq:precir}
P_q = \frac{RR_q}{RET}
\end{equation}

\begin{equation}\label{eq:preck}
P_q@k = \frac{RR^k_q}{k}
\end{equation}

\begin{equation}\label{eq:prec1}
MP@1 = \frac{1}{Q} \sum_q P_q@1
\end{equation}

In classification, precision is a measure of how many correct predictions
were made out of all predictions. Equation~\ref{eq:preccls} shows this
relationship: $TP$ are the true positives, the correct predictions,
and $PRED$ is the number of all predictions, including incorrect ones.
$PRED$ is usually equal to the size of the testing dataset in classification.

In information retrieval, similarly,
precision and more generally Precision$@k$ is a measure of how many
relevant documents were retrieved out of all retrieved documents.
However, the metric is evaluated for each query, instead of the whole
testing dataset.
Equations~\ref{eq:precir}~and~\ref{eq:preck} illustrate this: $P_q$ is
the precision for query $q$, $RR_q$ represents the number of retrieved
and relevant documents for query $q$.
$RET$ represents the total number of retrieved documents. Finally, $RR^k_q$
represents the number of retrieved, relevant documents out of the $k$ first
retrieved documents for query $q$.

Thus, Precision$@k$ simply uses a cut-off at $k$: it measures how many
relevant documents there are in the first $k$ retrieved documents.

In instance search, only the first retrieved document is important:
If the first result is correct, we correctly identify the instance.
Thus, we are mostly interested in Precision@1 in our research.
In order to generalize this metric to all queries, the mean Precision@1
is used, as defined in Equation~\ref{eq:prec1}. Here, $Q$ represents
the number of queries.

\subsection{Mean average precision}
\begin{equation}\label{eq:recall}
R_q = \frac{RR_q}{REL_q}
\end{equation}

\begin{equation}\label{eq:ap}
AP_q = \frac{1}{REL_q} \sum_{k=1}^{RET} rel(RET_k)P@k
\end{equation}

\begin{equation}\label{eq:map}
MAP = \frac{1}{Q} \sum_q AP_q
\end{equation}

In information retrieval, average precision is commonly used, since it
allows to consider the order in which documents are returned by a system.
Average precision represents the area under the precision-recall curve
for a given query $q$. Recall for query $q$ is defined as in
Equation~\ref{eq:recall} as the fraction of retrieved, relevant documents
for query $q$ as compared to all relevant documents to query $q$.

Equation~\ref{eq:ap} defines average precision for a given query $q$:
it can be expressed as a finite sum
over the Precision@k values at every possible cut-off $k$, multiplied by an
indicator $rel(RET_k)$, which is 1 if the $k$-th retrieved document is
relevant. This sum is then normalized by the total number of relevant
documents for query $q$ $REL_q$.

Finally, mean average precision $MAP$ represents the mean of all average
precision values over $Q$ queries, as defined in Equation~\ref{eq:map}.

In instance search, mean average precision is not important as we only
care about the first ranked result. Nevertheless, it can be an interesting
metric to allow better comparison between different systems, so we compute
the mean average precision for all results along with the mean Precision@1.

\section{Baselines}
Table %TODO
shows the baselines set for the two evaluation datasets, as described in
Section~\ref{sec:datasets}.

\section{Performance}
\subsection{Training}
\subsection{Evaluation}
% TODO talk about some performance metrics (how much time to train, to evaluate, depending on network. what hardware was used)
