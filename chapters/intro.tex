% !TeX root = ../full_report.tex

\chapter{Introduction}
\section{Motivation}
The research presented here is motivated by the Guimuteic project,
a collaborative project between industry and LIG. The
aim is to develop a smart audio-guide for touristic or cultural sites.
% TODO cite project/LIG/etc

In practice, the final product should offer an augmented reality
interface to the user, with information about the object or objects
the user is looking at.

One part of the development consists in finding ways of identifying
objects the user is looking at. There are multiple possibilities,
for example based on the geo-localization of the user and other sensors.
In our research, we focus on the recognition of objects based
only on visual clues.

\section{Research problem}
More specifically, we are interested in the following problem:
we are given a collection with reference images for each object,
or instance, to be recognized. The task is to develop a system that,
given an image of one of the instances, can decide which instance
the image represents. We assume that, on average, less than ten reference
images are available for each instance.
We will refer to this problem as instance retrieval in the following.

There are a few problems similar to instance retrieval. For one, there is
the image classification problem: we are given a collection of images
where each image is assigned a semantic class,
such as \emph{dog} or \emph{fridge}.
The task is to develop a system that, given an image, decides which
class the image represents.
This problem is of course similar to instance retrieval if we
consider each instance to be a separate class.
However, two main differences exist: in classification, the classes are
usually semantic classes and one of the goals is to find a pattern shared
by each semantic class. This is not as important when we consider a single
object to represent a class. A consequence of this is that in classification
we are usually given many reference images per class:
hundreds or even thousands.
This means our problem is closer to the few-shots classification problem,
where only a few images are available for each class.

Another similar problem is image retrieval. In image retrieval,
we are given a collection of reference images and a query image.
We aim to rank the reference images by similarity to the query image.
Usually, in image retrieval, reference images are not labeled or loosely
labeled and many images are irrelevant to the query image. The challenge
is to rank the most similar images on top.
Instance retrieval is related to image retrieval in that we aim to
develop a notion of similarity between images. However, in instance
retrieval, we do not care about the rank of the returned images, since
we only consider the highest ranked image, which should represent the
instance to retrieve.

% TODO possibly put this into different section
Finally, to allow fast evaluation of the system, our goal is to develop
a system that provides a single descriptor for each image, which should
be small enough to be easily stored for all reference images and fast
to compute.
Two images should be compared according to a distance metric, which should
also be fast to compute. This requirement is important, since a possible
device used as an augmented audio-guide will record many images per second.
Thus, to be responsive, it should be possible to evaluate the system in
real-time or close to real-time.

% TODO possibly elaborate here

\section{Challenges}
For all of the problems described above, deep learning approaches
based on convolutional neural networks (CNNs) have recently obtained
the state-of-the-art results. One of the drawbacks of deep CNNs is that
they require large amounts of data to be trained.

In our research problem in particular, we do not
have large enough amounts of data available to fully train a deep CNN.
However, we aim to develop a system that can learn a descriptor
specific to the reference images,
as the system only needs to recognize and differentiate between
instances of that particular dataset. Thus the system should be tuned to
the dataset. Since we usually have less than ten images per instance,
this is an important challenge to overcome.

Another challenge comes from the nature of the datasets: we were not
able to find similar datasets in the literature, with only a few,
clean images, for each instance and many instances.
Common datasets in image retrieval like Oxford5k, Paris6k or Holidays
~\cite{philbin_object_2007,philbin_lost_2008,jegou_hamming_2008}
contain many images per instance, as well as a lot of noise,
as they are used to evaluate whether a system can robustly retrieve the
correct images out of a set of weakly related images.
This means that the approaches used for image retrieval datasets may
not work as well for instance retrieval.

\section{Contributions}
We made the following three major contributions:
\begin{enumerate}
    \item We analyze existing approaches to image retrieval, classification
    and determine their shortcomings in our problem setting.
    \item Based on these shortcomings, we propose a novel approach
    to improve results in our problem setting. This approach is detailed
    in Section~\ref{sec:proposed}. It is based on
    extracting region proposals from images through a fully convolutional
    network (FCN), combined with the previous state-of-the-art architecture
    in image retrieval.
    \item We evaluate this novel approach and compare with previously
    proposed approaches.
\end{enumerate}
