% !TeX root = ../full_report.tex

\chapter{Conclusion}
\section{Conclusions from our research}
During our research, we analyzed state-of-the-art approaches in
image classification and image retrieval and their application to
the instance search problem. We found that these approaches struggle
with query images and reference images being at different scales.
Additionally, the region proposals used by the state-of-the-art do not
seem applicable to the problem of instance search.

We have presented a novel approach, which consists of two key steps.
First, we leverage the concept of fully convolutional networks in order
to perform classification training at different scales, without a heavy
computational overhead. Second, we show that the fully convolutional
network can be used to obtain region proposals without the need for
an additional component in the network and training. This is particularly
important, since region proposals are difficult to define manually in
our research problem.

Finally, the proposed network keeps all the benefits of state-of-the-art
approaches: it can be trained end-to-end and it produces an effective
global descriptor, which can be compared using the dot-product.
Additionally, it is modular in the sense that it can be built upon
any type of CNN, pre-trained for classification.
Furthermore, the proposed network is fast to evaluate and the training
time for each dataset is reasonable.

Through multiple experiments on two datasets, we show that the descriptor
obtained using our proposed network outperforms previous state-of-the-art
approaches on the instance search task, while being just as memory-efficient
and fast.

\section{Future work}
\subsection{Few-shots classification}
While we explored different ideas in this report, some have not been
considered yet. For one, a paper by Kaiser et al~\cite{kaiser_learning_2017}
was presented earlier this year, which focuses on few-shots classification
and achieves state-of-the-art results.

As noted before, our research problem shares many similarities with the
few-shots classification problem. Furthermore, this paper presents a
method of combining the triplet loss and cross-entropy loss in order to
organize a memory of so-called rare events. This is similar to our proposed
approach and may be adapted to our research problem. At the very least,
it could provide significant insight to the instance search problem.

\subsection{Conditional similarities}
Another interesting paper presented earlier this year is by
Veit et al~\cite{veit_conditional_2016}. It presents a method of
capturing semantically different notions of similarity. The main idea
is that there are types of similarity (color, category, etc.) and
thus images are embedded into many different subspaces, which capture
these different types of similarity. Finally, they present a method
to jointly learn a global embedding, made up of different sub-embeddings
for the sub-spaces representing different similarities.

This idea could be interesting in the instance search problem, as it
may allow us to learn different types of categories first and then produce
an embedding based on these different categories. For instance, a painting
could be described by its main tone (dark, bright, etc.), its art movement
(impressionism, dadaism, etc.) or simply its size (big, small, squared,
wide, high). Each of these notions may be easily described individually
and a joint learning of these different types of descriptors could provide
significant improvements over learning a single, global descriptor for
each image.
