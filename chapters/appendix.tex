% !TeX root = ../full_report.tex

\appendix \chapter{Appendix}

\section{Reproducibility}
The scripts used to produce all results presented in this work are accessible
publicly here: \url{https://github.com/MatthiasKohl/nnForRetrieval}

For training and testing on a new dataset, the following steps are needed:

\begin{enumerate}
    \item Required packages: Python 2.7, NumPy, SciPy, OpenCV-Python
    (Python bindings for OpenCV 3.2) and PyTorch 0.12
    \item Edit and run \emph{pre\_process\_dataset.py} to pre-process
    images to be at the correct size (or correct size on the smaller
    side and given maximal aspect ratio)
    \item Edit and run \emph{create\_mean\_std\_file.py} to create the
    file needed to normalize images by mean and standard deviation
    \item Edit \emph{utils/params.py} and add all global parameters
    to the following dictionaries. All dictionaries have the dataset
    ID as key. This is the first part of the file-name of the dataset
    (excluding folder path) before an underscore
    \begin{enumerate}
        \item \emph{mean\_std\_files}: key is dataset ID, value is the file-name
        of the file generated by \emph{create\_mean\_std\_file.py}
        \item \emph{match\_label\_functions}: a function to extract labels
        from the name of an image
        \item \emph{num\_classes}: the number of classes/instances in the dataset
        \item \emph{image\_sizes}: this is usually always (3, 224, 224)
    \end{enumerate}
    \item Run \emph{python test/[training method]\_test.py} and follow
    the instructions for command-line parameters, which should be self
    explanatory
    \item For training, edit \emph{train/[training method]\_p.py}.
    Most parameters should be self explanatory
    \item Run \emph{python train/[training method].py} from the main
    directory
\end{enumerate}

% TODO possibly add paper here
