% !TeX root = ../full_report.tex

\chapter{Contributions}
\section{Fine-tuning a CNN}
A first possible approach to this problem is to simply try to fine-tune
a classification network to instance classification. This means that
we consider each instance as a class, and optimize using a cross-entropy
loss, typical for classification.

\begin{equation}\label{eq:crossent}
\mathcal{L} = - \frac{1}{N}
\sum_{i=1}^N y_i \log \hat{y}_i + (1-y_i) \log (1-\hat{y}_i)
\end{equation}

Equation~\ref{eq:crossent} describes the cross-entropy loss for a
given instance. $N$ is the number of samples, $y_i$ an indicator
variable taking the value $1$ if sample $i$ belongs to the given
instance and $0$ otherwise, and $\hat{y}_i$ is the predicted probability
that sample $i$ belongs to the given instance.

Fine-tuning a CNN for classification on different data has been studied
intensively by Yosinki et al~\cite{yosinski_how_2014}. In particular,
their study shows that it is only important to fine-tune the neurons
of higher layers of a CNN. Furthermore, they show that
fine-tuning can increase generalization even in the fine-tuned model.
Both of these properties are desirable in our task, because they reduce
the memory and time needed to train a network, as well as the need
for a large dataset used for fine-tuning.

The specific considerations we take with regards to fine-tuning are
described in the following sections.

\subsection{Data augmentation}
Our datasets % TODO ref
contain an average of less than 10 samples per instance. This is too little
to train a typical CNN model designed for classification, even when
fine-tuning.
One way to overcome this is to augment the data, by randomly applying
affine transformations, color perturbations and other random transformations.
% TODO possibly reference some paper here

Since we specifically identified an issue related to different scales
in images, it makes sense to augment the data by randomly scaling the
images. % TODO ref issue scaling

We found that randomly rotating and flipping the images improved performance
as well.

\subsection{Transfer learning}
The modularity of a CNN means that we can easily transfer
the weights from a pre-trained model, and only retrain the highest
abstraction layers. Specifically, we re-train all linear layers in the
model, representing the highest-level layers.

We also re-train the highest level convolutional layers, since our datasets
contain many visually different images as compared to the ImageNet
dataset used for pre-training the models.
For the AlexNet architecture, we choose to re-train all layers above
and including the last convolutional layer.
For a ResNet architecture, we re-train all layers above and including the
third to last block of convolutional layers. This contains the
nine highest convolutional layers in total.

\section{Proposed approach}
