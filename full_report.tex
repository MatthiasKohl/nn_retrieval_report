\documentclass[12pt, a4paper]{memoir} % for a short document
\usepackage[french,english]{babel}

\usepackage [vscale=0.76,includehead]{geometry}                % See geometry.pdf to learn the layout options. There are lots.
%\geometry{a4paper}                   % ... or a4paper or a5paper or ...
%\geometry{landscape}                % Activate for for rotated page geometry
%\OnehalfSpacing
% \setSingleSpace{1.05}
%\usepackage[parfill]{parskip}    % Activate to begin paragraphs with an empty line rather than an indent


%===================================== packages
\usepackage{lipsum}
\usepackage{graphicx}
\usepackage{subcaption}
\usepackage{fullpage}
\usepackage{mathptmx} % font = times
\usepackage{helvet} % font sf = helvetica
\usepackage[utf8]{inputenc}
\usepackage{relsize}
\usepackage[T1]{fontenc}
\usepackage{tikz}
\usepackage{booktabs}
\usepackage{textcomp}%textquotesingle
\usepackage{multirow}
\usepackage{pgfplots}
\usepackage{url}
\usepackage{footnote}
\usepackage{amsmath,amssymb}
\usepackage{hyperref}
\usepackage{colortbl}
%============================================
\usetikzlibrary{arrows,shapes,positioning,shadows,trees}
\makesavenoteenv{tabular}
\makesavenoteenv{table}
%==============================================
\def\checkmark{\tikz\fill[scale=0.4](0,.35) -- (.25,0) -- (1,.7) -- (.25,.15) -- cycle;}
%Style des têtes de section, headings, chapitre
\headstyles{komalike}
\nouppercaseheads
\chapterstyle{dash}
\makeevenhead{headings}{\sffamily\thepage}{}{\sffamily\leftmark}
\makeoddhead{headings}{\sffamily\rightmark}{}{\sffamily\thepage}
\makeoddfoot{plain}{}{}{} % Pages chapitre.
\makeheadrule{headings}{\textwidth}{\normalrulethickness}
%\renewcommand{\leftmark}{\thechapter ---}
\renewcommand{\chaptername}{\relax}
\renewcommand{\chaptitlefont}{ \sffamily\bfseries \LARGE}
\renewcommand{\chapnumfont}{ \sffamily\bfseries \LARGE}
\setsecnumdepth{subsection}


% Title page formatting -- do not change!
\pretitle{\HUGE\sffamily \bfseries\begin{center}}
\posttitle{\end{center}}
\preauthor{\LARGE  \sffamily \bfseries\begin{center}}
\postauthor{\par\end{center}}
\newcommand{\jury}[1]{%
\gdef\juryB{#1}}
\newcommand{\juryB}{}
\newcommand{\session}[1]{%
\gdef\sessionB{#1}}
\newcommand{\sessionB}{}
\newcommand{\option}[1]{%
\gdef\optionB{#1}}
\newcommand{\optionB} {}

\renewcommand{\maketitlehookd}{%
\vfill{}  \large\par\noindent
\begin{center}\juryB \bigskip\sessionB\end{center}
\vspace{-1.5cm}}
\renewcommand{\maketitlehooka}{%
\vspace{-1.5cm}\noindent\includegraphics[height=12ex]{./img/logo-uga.png}\hfill\raisebox{2ex}{\includegraphics[height=14ex]{./img/logoINP.png}}\\
\bigskip
\begin{center} \large
Master of Science in Informatics at Grenoble \\
Master Informatique \\
Specialization \optionB  \end{center}\vfill}
% =======================End of title page formatting

\option{Data Science}
\title{Deep learning with Siamese networks for instance search or identification} %\\\vspace{-1ex}\rule{10ex}{0.5pt} \\sub-title}
\author{Matthias Kohl}
\date{June 20, 2017} % Delete this line to display the current date
\jury{
Research project performed at LIG - MRIM\\\medskip
Under the supervision of:\\
Georges Quénot, Jean-Pierre Chevallet\\\medskip
% Defended before a jury composed of:\\
% Head of the jury (TODO)\\
% Jury member 1\\
% Jury member 2\\
}
\session{June \hfill 2017}
\setcounter{tocdepth}{4}
\setcounter{secnumdepth}{4}

%%% BEGIN DOCUMENT
\begin{document}
\selectlanguage{English} % french si rapport en français
\frontmatter
\begin{titlingpage}
\maketitle
\end{titlingpage}

%\small
\setlength{\parskip}{-1pt plus 1pt}

\renewcommand{\abstracttextfont}{\normalfont}
\abstractintoc
\begin{abstract}
Our research is part of the GUIMUTEIC project, which aims at building an augmented
audio-guide for museums or tourist sites. In particular, our goal is to
develop a system capable of identifying a
piece of art using solely
the visual clues of an image provided by a camera, in order to provide
further information about the piece of art to the user of the audio-guide.
We focus on instance retrieval approaches based on deep learning.
We make three main contributions:
we analyze advantages and limitations of the current state of the art
in instance retrieval,
we propose a novel approach based on a fully
convolutional network (FCN),
and we evaluate the performance of these different approaches in depth.
We show that the proposed architecture produces state-of-the-art results,
is fast to run and reasonably fast to setup for a new museum or
tourist site. The evaluation is carried out on the CLICIDE and GaRoFou
datasets, both created in the context of the project, containing images
of pieces of art from museums in Grenoble and Lyon.
\end{abstract}
\abstractintoc

\renewcommand\abstractname{Acknowledgement}
\begin{abstract}
This work has been carried out in the context of the Guimuteic project
funded by Fonds Européen de Développement Régional (FEDER) of région
Auvergne Rhône-Alpes.

I would like to express my sincere gratitude to Georges Quénot,
Jean-Pierre Chevallet and Maxime Portaz for their invaluable assistance
and comments in reviewing this report.
\end{abstract}

\renewcommand\abstractname{Publication}
\begin{abstract}
The work developed as part of this research project will be submitted as a
paper to the second international workshop on egocentric perception,
interaction and computing at ICCV 2017.
\end{abstract}

\renewcommand\abstractname{R\'esum\'e}
\begin{abstract} \selectlanguage{French}
Notre travail de recherche fait partie du projet GUIMUTEIC qui doit produire un audio-guide
augment\'e pour des mus\'ees ou des sites touristiques. En particulier, notre objectif est
de d\'evelopper un système d'identification automatique de l'\oe{}uvre d'art
contempl\'e par un utilisateur de l'audio-guide, en utilisant uniquement
une image issue d'une caméra.
Notre recherche se focalise sur les approches d'apprentissage profond
pour l'identification d'instances.
Nous apportons trois contributions majeures:
une analyse des avantages et inconv\'enients
des solutions d'identification d'instances de l'\'etat de l'art, la proposition d'une nouvelle approche bas\'ee sur
un r\'eseau enti\`erement convolutionnel (FCN),
et l'\'evaluation de la performance de ces diff\'erentes
approches. Nous montrons ainsi que l'architecture propos\'ee donne des
r\'esultats performants tout en \'etant rapide \`a l'interrogation et raisonnablement rapide \`a mettre en place pour une nouvelle collection de mus\'ee ou site touristique. L'\'evaluation se fait sur les deux corpus d'images CLICIDE et GaRoFou, tous deux produits dans le cadre du projet, et contenant des images d'\oe{}uvres d'art d'un mus\'ee de Grenoble et de Lyon.
\end{abstract}
\selectlanguage{English}

\cleardoublepage

\tableofcontents* % the asterisk means that the table of contents itself isn't put into the ToC
\normalsize

\mainmatter
\SingleSpace
%==============================CHAPTERS==================
% !TeX root = ../full_report.tex

\chapter{Introduction}
\section{Motivation}
The research presented here is motivated by the Guimuteic project,
a collaborative project between industry and LIG. The
aim is to develop a smart audio-guide for touristic or cultural sites.
% TODO cite project/LIG/etc

In practice, the final product should offer an augmented reality
interface to the user, with information about the object or objects
the user is looking at.

One part of the development consists in finding ways of identifying
objects the user is looking at. There are multiple possibilities,
for example based on the geo-localization of the user and other sensors.
In our research, we focus on the recognition of objects based
only on visual clues.

\section{Research problem}
More specifically, we are interested in the following problem:
we are given a collection with reference images for each object,
or instance, to be recognized. The task is to develop a system that,
given an image of one of the instances, can decide which instance
the image represents. We assume that, on average, less than ten reference
images are available for each instance.
We will refer to this problem as instance retrieval in the following.

There are a few problems similar to instance retrieval. For one, there is
the image classification problem: we are given a collection of images
where each image is assigned a semantic class,
such as \emph{dog} or \emph{fridge}.
The task is to develop a system that, given an image, decides which
class the image represents.
This problem is of course similar to instance retrieval if we
consider each instance to be a separate class.
However, two main differences exist: in classification, the classes are
usually semantic classes and one of the goals is to find a pattern shared
by each semantic class. This is not as important when we consider a single
object to represent a class. A consequence of this is that in classification
we are usually given many reference images per class:
hundreds or even thousands.
This means our problem is closer to the few-shots classification problem,
where only a few images are available for each class.

Another similar problem is image retrieval. In image retrieval,
we are given a collection of reference images and a query image.
We aim to rank the reference images by similarity to the query image.
Usually, in image retrieval, reference images are not labeled or loosely
labeled and many images are irrelevant to the query image. The challenge
is to rank the most similar images on top.
Instance retrieval is related to image retrieval in that we aim to
develop a notion of similarity between images. However, in instance
retrieval, we do not care about the rank of the returned images, since
we only consider the highest ranked image, which should represent the
instance to retrieve.

% TODO possibly put this into different section
Finally, to allow fast evaluation of the system, our goal is to develop
a system that provides a single descriptor for each image, which should
be small enough to be easily stored for all reference images and fast
to compute.
Two images should be compared according to a distance metric, which should
also be fast to compute. This requirement is important, since a possible
device used as an augmented audio-guide will record many images per second.
Thus, to be responsive, it should be possible to evaluate the system in
real-time or close to real-time.

% TODO possibly elaborate here

\section{Challenges}
For all of the problems described above, deep learning approaches
based on convolutional neural networks (CNNs) have recently obtained
the state-of-the-art results. One of the drawbacks of deep CNNs is that
they require large amounts of data to be trained.

In our research problem in particular, we do not
have large enough amounts of data available to fully train a deep CNN.
However, we aim to develop a system that can learn a descriptor
specific to the reference images,
as the system only needs to recognize and differentiate between
instances of that particular dataset. Thus the system should be tuned to
the dataset. Since we usually have less than ten images per instance,
this is an important challenge to overcome.

Another challenge comes from the nature of the datasets: we were not
able to find similar datasets in the literature, with only a few,
clean images, for each instance and many instances.
Common datasets in image retrieval like Oxford5k, Paris6k or Holidays
~\cite{philbin_object_2007,philbin_lost_2008,jegou_hamming_2008}
contain many images per instance, as well as a lot of noise,
as they are used to evaluate whether a system can robustly retrieve the
correct images out of a set of weakly related images.
This means that the approaches used for image retrieval datasets may
not work as well for instance retrieval.

\section{Contributions}
We made the following three major contributions:
\begin{enumerate}
    \item We analyze existing approaches to image retrieval, classification
    and determine their shortcomings in our problem setting.
    \item Based on these shortcomings, we propose a novel approach
    to improve results in our problem setting. This approach is detailed
    in Section~\ref{sec:proposed}. It is based on
    extracting region proposals from images through a fully convolutional
    network (FCN), combined with the previous state-of-the-art architecture
    in image retrieval.
    \item We evaluate this novel approach and compare with previously
    proposed approaches.
\end{enumerate}

% !TeX root = ../full_report.tex

\chapter{State of the art}
\section{SIFT and bag-of-words}
Until recently, the state of the art in image retrieval and matching
was based on the idea of
bag-of-words~\cite{philbin_object_2007}~\cite{mikulik_learning_2013}.
The idea behind the approach is to represent an image as a histogram,
or collection of frequencies, of visual words. The visual words should
be small, representative patches of the images in the dataset.
Usually, these visual words are obtained in multiple steps. First,
local features are extracted and encoded. The most common feature used
is SIFT~\cite{TODO}, a 128-dimensional vector representing
scale-invariant features. Then, the features are extracted for all
images and clustered into clusters of similar features.
The representative for each cluster is the center of the cluster,
i.e. the mean of all features falling into that cluster.
Each image can then be represented as a histogram of the occurrences
of these representative features.
This histogram forms the image descriptor, which has as many dimensions
as there are representative features and clusters.

Finally, for image classification, a classifier can be learned from the
descriptors of all images in the dataset. On the other hand, for image
retrieval or matching, the descriptors can be directly matched against
each other, based on some similarity measure.
In both cases, it can be useful to project the descriptors into a Hilbert
space using a different similarity measure than euclidean distance
in the original descriptor space.
For classification, this is done by employing an SVM classifier with a
non-linear kernel~\cite{TODO}. Among the most popular kernels are the
radial basis function~\cite{} and the chi-squared function~\cite{}.

Until recently, this approach has obtained state of the art results for
image retrieval tasks~\cite{mikulik_learning_2013}.

\section{Deep learning with CNNs}
Starting with the results of AlexNet for image classification in the 2012
ImageNet challenge~\cite{krizhevsky_imagenet_2012}~\cite{TODO},
image classification tasks have been dominated by CNNs, learned using
large amounts of data.

A general trend in image related tasks is to move to an end-to-end
approach, where the final objective is directly optimized using gradient
descent and the gradient is back-propagated to all previous parts of the
system. In contrast, the bag-of-words model requires a choice of
features (SIFT, ORB (TODO mention ORB before), \dots),
a choice of the method for clustering features (k-means),
a choice of the classifier (AdaBoost~\cite{}, SVM, \dots),
as well as a choice of the kernel if an SVM classifier is used.

Using a CNN, features are extracted at a low abstraction level by the
first convolutional layers, then higher level features are formed
by combining low level features from the previous layers. Finally,
high-level features are combined into a classifier by linear layers.
The advantage of this approach is that the features are learned at all
abstraction levels. Furthermore, the modularity of the approach allows
us to easily transfer the lower level features learned from a large dataset
to a smaller dataset, where there may not be enough data to efficiently
learn lower level features.

After AlexNet, many improvements to the CNN architecture have been
suggested. First, the VGG architecture uses smaller convolutional kernels
and thus more layers to reduce the number of parameters and increase
the non-linearity of the network~\cite{simonyan_very_2014}. Second,
even deeper networks were suggested. However, a very deep network
is hard to train because of the vanishing gradient problem~\cite{}.
This problem is overcome in the ResNet~\cite{he_deep_2015} and
Inception~\cite{szegedy_inception-v4_2016} architectures by introducing
skip connections. A skip connection means that, for a block of two or
three layers, the output of the previous layer is added to the output
of the block. In the Inception architecture, skip connections of different
lengths are combined. Finally, the DenseNet~\cite{} architecture takes
the skip connection idea one step further: each layer is dependent on
the output of all $n$ previous layers.

\section{Image retrieval using CNNs}
For image retrieval, the current state of the art set by
Gordo et al~\cite{gordo_deep_2016} is based on such an
end-to-end approach. The goal is to learn a global descriptor for images
that is well suited for comparing images.
This is achieved by extracting the convolutional features of a pre-trained
CNN. Then, a Region Proposal Network (RPN)~\cite{TODO faster rcnn} is
used to extract the regions of interest. For each region of interest,
a shifting and linear layer are used to reduce the dimensionality of
the descriptor. The final descriptor is simply a normalized sum of the
region-wise descriptors. This network can be learned end-to-end and
the obtained descriptor achieves state of the art results, which can
be even further improved by using query expansion and database side
feature augmentation (TODO either remove or mention before).


% !TeX root = ../full_report.tex

\chapter{Contributions}
\section{Fine-tuning a CNN}\label{sec:finetuning}
A first possible approach to this problem is to simply try to fine-tune
a classification network to instance classification. This means that
we consider each instance as a class, and optimize using a cross-entropy
loss, typical for classification.

\begin{equation}\label{eq:crossent}
\mathcal{L} = - \frac{1}{N}
\sum_{i=1}^N y_i \log \hat{y}_i + (1-y_i) \log (1-\hat{y}_i)
\end{equation}

Equation~\ref{eq:crossent} describes the cross-entropy loss for a
given instance. $N$ is the number of samples, $y_i$ an indicator
variable taking the value $1$ if sample $i$ belongs to the given
instance and $0$ otherwise, and $\hat{y}_i$ is the predicted probability
that sample $i$ belongs to the given instance.

Fine-tuning a CNN for classification on different data has been studied
intensively by Yosinki et al~\cite{yosinski_how_2014}. In particular,
their study shows that it is only important to fine-tune the neurons
of higher layers of a CNN. Furthermore, they show that
fine-tuning can increase generalization even in the fine-tuned model.
Both of these properties are desirable in our task, because they reduce
the memory and time needed to train a network, as well as the need
for a large dataset used for fine-tuning.

The specific considerations we take with regards to fine-tuning are
described in the following sections.

\subsection{Data augmentation}
Our datasets % TODO ref
contain an average of less than 10 samples per instance. This is too little
to train a typical CNN model designed for classification, even when
fine-tuning.
One way to overcome this is to augment the data, by randomly applying
affine transformations, color perturbations and other random transformations.
% TODO possibly reference some paper here

Since we specifically identified an issue related to different scales
in images, it makes sense to augment the data by randomly scaling the
images. % TODO ref issue scaling

We found that randomly rotating and flipping the images improved performance
as well.

\subsection{Transfer learning}
The modularity of a CNN means that we can easily transfer
the weights from a pre-trained model, and only retrain the highest
abstraction layers. Specifically, we re-train all linear layers in the
model, representing the highest-level layers.

We also re-train the highest level convolutional layers, since our datasets
contain many visually different images as compared to the ImageNet
dataset used for pre-training the models.
For the AlexNet architecture, we choose to re-train all layers above
and including the last convolutional layer.
For a ResNet architecture, we re-train all layers above and including the
third to last block of convolutional layers. This contains the
nine highest convolutional layers in total.

\subsection{Analysis of previous approaches}\label{sec:analysisprev}
\subsubsection{Identifying regions of interest}
Previous approaches in image retrieval~\cite{gordo_end--end_2016}
~\cite{salvador_faster_2016}~\cite{tolias_particular_2015}
usually deal with regions of interest in one way or another.
The idea is that in most cases, only certain parts of each image can
be useful for comparison with other images. In addition to this,
cropping images at their regions of interest can help with differences
in scale of the images to compare: if a building is visible only in a small
part of an image, cropping the image at that part and then re-scaling
the part should set the building at a normalized scale.

However, in instance search with museum datasets, it is not obvious
where the regions of interest should be: most images represent an entire
painting or parts of it and only some may contain the painting as part
of the image with a wall in the background. This means for most images,
the ground-truth region of interest is simply the entire image, and some
may have a ground-truth region of interest which is almost the entire image,
excluding only a small part of the background.

On the other hand, a network fine-tuned on classification on such a dataset
should be able to easily identify the region containing the painting, since
the background wall is contained in almost all classes, which means it is a
particularly bad indicator of the class. Thus, if the network is applied in
a strided manner across an image, it should produce low maximal activations
in parts containing big sections of background wall.

\begin{figure}
\centering
\begin{subfigure}[b]{0.3\textwidth}
\includegraphics[width=\textwidth]{img/sample1_10A-0519.png}
\caption{Image with label 10A\label{fig:sample1_id}}
\end{subfigure}
\begin{subfigure}[b]{0.3\textwidth}
\includegraphics[width=\textwidth]{img/sample1_heatmap.png}
\caption{Heat-map for 10A\label{fig:sample1_hm}}
\end{subfigure}
\begin{subfigure}[b]{0.3\textwidth}
\includegraphics[width=\textwidth]{img/sample1_labels.png}
\caption{Label-map for 10A\label{fig:sample1_lab}}
\end{subfigure}

\begin{subfigure}[b]{0.3\textwidth}
\includegraphics[width=\textwidth]{img/sample2_5P-0508.png}
\caption{Image with label 5P\label{fig:sample2_id}}
\end{subfigure}
\begin{subfigure}[b]{0.3\textwidth}
\includegraphics[width=\textwidth]{img/sample2_heatmap.png}
\caption{Heat-map for 5P\label{fig:sample2_hm}}
\end{subfigure}
\begin{subfigure}[b]{0.3\textwidth}
\includegraphics[width=\textwidth]{img/sample2_labels.png}
\caption{Label-map for 5P\label{fig:sample2_lab}}
\end{subfigure}

\begin{subfigure}[b]{0.3\textwidth}
\includegraphics[width=\textwidth]{img/sample3_30P-0976.png}
\caption{Image with label 30P\label{fig:sample3_id}}
\end{subfigure}
\begin{subfigure}[b]{0.3\textwidth}
\includegraphics[width=\textwidth]{img/sample3_heatmap.png}
\caption{Heat-map for 30P\label{fig:sample3_hm}}
\end{subfigure}
\begin{subfigure}[b]{0.3\textwidth}
\includegraphics[width=\textwidth]{img/sample3_labels.png}
\caption{Label-map for 30P\label{fig:sample3_lab}}
\end{subfigure}
\caption{Sample images (scaled to a smaller side of 448 pixels)
along with the heat-map of maximal activation values
at each location when a fine-tuned ResNet-152 is applied to the image in a
strided manner, as well as the labels of all maximal activations that are
greater than the mean maximal activation\label{fig:heatmaps}}
\end{figure}

Figure~\ref{fig:heatmaps} shows images, along with the heat map
representing the maximal activation of a fine-tuned ResNet-152 at
each coordinate, when the network is applied in a strided manner
across the input image. The fine-tuning was carried out as described in
Section~\ref{sec:finetuning}.
From this image, we can see that the highest maximal
activations of the network usually occur at the location of the object.
This is true even if the object is not correctly classified by some of
the highest activations as can be seen in images
~\ref{fig:sample2_id}~-~\ref{fig:sample2_lab}.

In the images
~\ref{fig:sample3_id}~-~\ref{fig:sample3_lab}, it seems like many
high maximal activations occur specifically in the background area.
However, the corresponding label-map shows that these areas correspond
to the labels 38E and 43D. Both of these labels are pieces of art which
consist mostly of the background wall. In this sense, it is not
entirely wrong to consider 'wall-only' patches of the image as instances of
these pieces of art. This simply means that the image consists of two
separate regions of interest: one region with the painting (label 30P)
and one region with the wall (labels 38E/43D).

From these observations, we can confirm the assumption that the
maximal activations of a fine-tuned network are a good indicator of
the location of an object, or a combination of different objects.
Using this assumption, there is no need
for a procedure to annotate regions of interest, as employed by most
image retrieval approaches. % TODO

On the other hand, using datasets developed for image retrieval,
such as Paris6k~\cite{philbin_lost_2008} or
Oxford5k~\cite{philbin_object_2007},
this assumption cannot be applied, since the dataset is not clean
enough for a fine-tuned network to be a good indicator of location of
the query objects.

\subsubsection{Incorrectly identified images}
\begin{figure}
\centering
\begin{subfigure}{\textwidth}
\begin{tabular}{|c|*{6}{c}}
\includegraphics[width=0.12\textwidth]{img/11J-0521.JPG} &
\includegraphics[width=0.12\textwidth]{img/11J-0.JPG} &
\includegraphics[width=0.12\textwidth]{img/11J-1.JPG} &
\includegraphics[width=0.12\textwidth]{img/11J-2.JPG} &
\includegraphics[width=0.12\textwidth]{img/11J-3.JPG} &
\includegraphics[width=0.12\textwidth]{img/11J-4.JPG} \\
\end{tabular}
\caption{Correctly identified query with label 11J,
along with its reference images.\newline
Average precision Gordo's net: 1.0, fine-tuned ResNet-152: 1.0
\label{fig:correct11J}}
\end{subfigure}

\begin{subfigure}{\textwidth}
\begin{tabular}{|c|*{6}{c}}
\includegraphics[width=0.12\textwidth]{img/23D-0740.JPG} &
\includegraphics[width=0.12\textwidth]{img/23D-0.JPG} &
\includegraphics[width=0.12\textwidth]{img/23D-1.JPG} &
\includegraphics[width=0.12\textwidth]{img/23D-2.JPG} &
\includegraphics[width=0.12\textwidth]{img/23D-3.JPG} &
\includegraphics[width=0.12\textwidth]{img/23D-4.JPG} &
\includegraphics[width=0.12\textwidth]{img/23D-5.JPG} \\
\end{tabular}
\caption{Correctly identified query with label 23D,
along with its reference images.\newline
Average precision Gordo's net: 1.0, fine-tuned ResNet-152: 1.0
\label{fig:correct23D}}
\end{subfigure}

\begin{subfigure}{\textwidth}
\begin{tabular}{|c|*{6}{c}}
\includegraphics[width=0.12\textwidth]{img/1C-0454.JPG} &
\includegraphics[width=0.12\textwidth]{img/1C-0.JPG} &
\includegraphics[width=0.12\textwidth]{img/1C-1.JPG} &
\includegraphics[width=0.12\textwidth]{img/1C-2.JPG} &
\includegraphics[width=0.12\textwidth]{img/1C-3.JPG} &
\includegraphics[width=0.12\textwidth]{img/1C-4.JPG} \\
\end{tabular}
\caption{Incorrectly identified query with label 1C,
along with its reference images.\newline
Average precision Gordo's net: 0.113, fine-tuned ResNet-152: 0.014
\label{fig:incorrect1C}}
\end{subfigure}

\begin{subfigure}{\textwidth}
\begin{tabular}{|c|*{6}{c}}
\includegraphics[width=0.12\textwidth]{img/5B-0506.JPG} &
\includegraphics[width=0.12\textwidth]{img/5B-0.JPG} &
\includegraphics[width=0.12\textwidth]{img/5B-1.JPG} &
\includegraphics[width=0.12\textwidth]{img/5B-2.JPG} &
\includegraphics[width=0.12\textwidth]{img/5B-3.JPG} &
\includegraphics[width=0.12\textwidth]{img/5B-4.JPG} \\
\end{tabular}
\caption{Incorrectly identified query with label 5B,
along with its reference images.\newline
Average precision Gordo's net: 0.013, fine-tuned ResNet-152: 0.002
\label{fig:incorrect5B}}
\end{subfigure}

\begin{subfigure}{\textwidth}
\begin{tabular}{|c|*{6}{c}}
\includegraphics[width=0.12\textwidth]{img/11C-0351.JPG} &
\includegraphics[width=0.12\textwidth]{img/11C-0.JPG} &
\includegraphics[width=0.12\textwidth]{img/11C-1.JPG} &
\includegraphics[width=0.12\textwidth]{img/11C-2.JPG} &
\includegraphics[width=0.12\textwidth]{img/11C-3.JPG} &
\includegraphics[width=0.12\textwidth]{img/11C-4.JPG} \\
\end{tabular}
\caption{Incorrectly identified query with label 11C,
along with its reference images.\newline
Average precision Gordo's net: 0.004, fine-tuned ResNet-152: 0.004
\label{fig:incorrect11C}}
\end{subfigure}
\caption{Sample images that were correctly and incorrectly identified
by a fine-tuned ResNet-152, as well as the network published by Gordo
et al~\cite{gordo_deep_2016}, referred to as Gordo's net. For each image,
the query image is shown at the very left and all possible correct reference
images are shown to the right of the query. The caption contains the
associated average precision values.
\label{fig:incorrectimg}}
\end{figure}

Figure~\ref{fig:incorrectimg}
shows typical images that were correctly identified versus images that
were incorrectly identified, by a fine-tuned network as well as the network
published by Gordo et al~\cite{gordo_deep_2016}, along with the average
precision for the respective queries.

From these results, we see that the networks struggle with images at
very different scales, achieving very low average precision,
while images with similar scales are usually perfectly matched.
% TODO more detail
Combining both of the properties identified in this section, we
propose a novel approach to learn a strong descriptor for instance
search in the next section.

\section{Proposed approach}
The proposed approach is based on multiple steps as described in the
following sections.

\subsection{Fine-tuning on classification using an FCN}\label{sec:fcnfinetune}
As shown in Section~\ref{sec:analysisprev}, a fine-tuned CNN is already
a good indicator of the location of an object in our datasets.
Additionally, it seems like scale is a particularly important factor.

Thus, the idea is to start by fine-tuning a network with images
at different scales. This can be achieved by using a fully
convolutional network (FCN), as introduced by
Long et al~\cite{long_fully_2015}.

In a FCN, the final fully connected layers
of a network are replaced by convolutional layers having a kernel
which fits the entire domain of the output of the previous layer.
This type of convolution is equivalent to a fully connected layer,
but allows inputs (and outputs) of any size.
The effect is that the network can be applied in one pass to an
arbitrarily sized image. The output then represents the activations
of the network as if it was applied in a strided manner across the image.
The stride of a full network depends on the architecture and is 32
pixels for the architectures used here: AlexNet and ResNet.

Once a FCN is applied to the image, the loss
is calculated by averaging the cross-entropy loss across all locations
and outputs.
The final loss is then obtained by passing images at different scales
through the FCN and averaging across all cross-entropy losses of all
outputs and scales.

When training, all images are normalized at scale to have the same
number of pixels in the smaller side. Note that for large aspect
ratios and large scales of the smaller side,
the memory consumption of training can be high for single images
having a very large aspect ratio. To limit this spike in memory
consumption, the aspect ratios are limited by introducing uniform
random noise on the smaller side of images with high aspect ratios.

\subsection{Descriptor extraction network}
\begin{figure}
\begin{subfigure}{\textwidth}
\includegraphics[width=\textwidth]{img/contrib_deploy.png}
\caption{Proposed architecture for instance search, at deploy time
\label{fig:contribdeploy}}
\end{subfigure}

\begin{subfigure}{\textwidth}
\includegraphics[width=\textwidth]{img/contrib_train.png}
\caption{Proposed architecture for instance search, at training time
\label{fig:contribtrain}}
\end{subfigure}
\caption{Proposed architecture for instance search, based on a FCN
~\cite{long_fully_2015} for region proposals. The descriptor extraction
for each region is similar to the architecture by
Gordo et al~\cite{gordo_deep_2016}, as shown in
Figures~\ref{fig:gordo_deploy}~and~\ref{fig:gordo_train}
\label{fig:contrib}}
\end{figure}

The second step of the proposed approach relies on the FCN, trained
as described in the first step in Section~\ref{sec:fcnfinetune}.

Figure~\ref{fig:contrib} illustrates the proposed architecture.
To obtain a descriptor, we first apply the convolutional layers of
a previous architecture, such as AlexNet or ResNet. We then obtain all
classification outputs at all locations using an FCN. We only consider the
maximal activation at all locations. The locations with the
top $k$ maximal activations will form the descriptor.

For each of these locations, we closely follow the architecture
proposed by Gordo et al~\cite{gordo_deep_2016}, as shown in
Figure~\ref{fig:gordo_deploy}. The convolutional features are
reduced by a L2-normalization, then a shifting and fully connected
layer. Finally, all descriptors from the $k$ locations are
sum-aggregated and L2-normalized again.

When training, the network is applied to a triplet of images.
All three descriptors are extracted and the triplet loss is
applied. To make sure that the locations with highest
maximal locations are correctly classified, a
cross-entropy loss is applied to each of the $k$ locations
used to compute the descriptor of the anchor image.
These cross-entropy losses are averaged over the $k$
locations.

\subsection{Advantages}
As shown in Section~\ref{sec:analysisprev}, this approach
allows the network to decide which region of interest is
best suited for classification and ultimately which regions
are best suited for comparison with other images.

Another advantage is that this approach does not require
any annotation of the images with regions of interest,
which can be a long, manual or automatic process, as evident
from the cleaning process used by Gordo et al~\cite{gordo_end--end_2016}.

Finally, an important property of the descriptor is that it
heavily relies on the classification
capabilities of the network. This means the descriptor is
mostly meaningless for a different dataset and needs
to be learned for each dataset. This can be an advantage,
since the descriptor can be better suited to a particular
dataset and the learning process does not take long, as shown
in Section~\ref{sec:perfresults}. On the other hand, this means that
the descriptor cannot be applied in a typical image retrieval
task.

% !TeX root = ../full_report.tex

\chapter{Evaluation}
\section{Datasets}\label{sec:datasets}
The proposed approaches as well as several base-lines are evaluated
on two datasets: the CLICIDE and GaRoFou datasets. These datasets are
described in detail by Portaz et al~\cite{portaz_construction_nodate}.
Both datasets are typical of instance search datasets
in museums or touristic sites: the objects represented by their images
are paintings for one and glass cabinets containing sculptures and artifacts
for the other dataset. Both datasets contain a small number of images
per instance and a small number of images in total.
Table~\ref{tab:datasets} lists the content and statistics of the datasets
in detail.

\subsection{CLICIDE}
The CLICIDE dataset contains photographs taken in the Grenoble Museum of Art.
The objects represented in the photographs are paintings and still images,
exclusively. The full dataset contains 207 images with no meaningful content,
mostly showing walls. These images are labeled as such and have been filtered
from the dataset in our evaluation. This explains the difference between
total number of images and number of images in Table~\ref{tab:datasets}.
Furthermore, there are 12 query images which contain objects not
present in the reference images. These have been filtered out as well.

The CLICIDE dataset is characteristic because the different images
for each instance usually consist of at least one global view of the
painting and multiple other images representing sub-regions of the
same painting.

\subsection{GaRoFou}
The GaRoFou dataset contains high-resolution photographs taken in the
Museun of Fourvière in Lyon. The objects represented in the photographs
are a cultural heritage collection: sculptures, steles and small artifacts
grouped together in glass cabinets.

The full GaRoFou dataset is larger than the part used here: along with
the dataset of images, it contains a dataset of videos as well as images
extracted from these videos. The video collection has not been used as
part of our research.

\begin{table}
\begin{tabular}{|*{3}{l|}}
\hline & \emph{CLICIDE} & \emph{GaRoFou}\\
\hline \emph{Type of objects} & Paintings & Glass cabinets
containing various objects\\
\hline \emph{\#Instances} & 464 & 311\\
\hline \emph{\#Reference images} & 3245 & 1068\\
\hline \emph{\#Query images (\#instances)} & 165 (134) & 184 (166)\\
\hline \emph{Median \#images per instance} & 6 & 2\\
\hline \emph{Min \#images per instance} & 1 & 1\\
\hline \emph{Max \#images per instance} & 22 & 45\\
\hline \emph{Total reference images (\#instances)} & 3452 (473) & 1068 (311)\\
\hline \emph{Total query images (\#instances)} & 177 (143) & 184 (166)\\
\hline
\end{tabular}
\caption{Details of the datasets used for evaluation\label{tab:datasets}}
\end{table}

\section{Metrics}
There are several commonly used metrics in classification and information
retrieval. The following sections define these metrics and their use
in our research.

\subsection{Precision}
\begin{equation}\label{eq:preccls}
P = \frac{TP}{PRED}
\end{equation}

\begin{equation}\label{eq:precir}
P_q = \frac{RR_q}{RET}
\end{equation}

\begin{equation}\label{eq:preck}
P_q@k = \frac{RR^k_q}{k}
\end{equation}

\begin{equation}\label{eq:prec1}
MP@1 = \frac{1}{Q} \sum_q P_q@1
\end{equation}

In classification, precision is a measure of how many correct predictions
were made out of all predictions. Equation~\ref{eq:preccls} shows this
relationship: $TP$ are the true positives, the correct predictions,
and $PRED$ is the number of all predictions, including incorrect ones.
$PRED$ is usually equal to the size of the testing dataset in classification.

In information retrieval, similarly,
precision and more generally Precision$@k$ is a measure of how many
relevant documents were retrieved out of all retrieved documents.
However, the metric is evaluated for each query, instead of the whole
testing dataset.
Equations~\ref{eq:precir}~and~\ref{eq:preck} illustrate this: $P_q$ is
the precision for query $q$, $RR_q$ represents the number of retrieved
and relevant documents for query $q$.
$RET$ represents the total number of retrieved documents. Finally, $RR^k_q$
represents the number of retrieved, relevant documents out of the $k$ first
retrieved documents for query $q$.

Thus, Precision$@k$ simply uses a cut-off at $k$: it measures how many
relevant documents there are in the first $k$ retrieved documents.

In instance search, only the first retrieved document is important:
If the first result is correct, we correctly identify the instance.
Thus, we are mostly interested in Precision@1 in our research.
In order to generalize this metric to all queries, the mean Precision@1
is used, as defined in Equation~\ref{eq:prec1}. Here, $Q$ represents
the number of queries.

\subsection{Mean average precision}
\begin{equation}\label{eq:recall}
R_q = \frac{RR_q}{REL_q}
\end{equation}

\begin{equation}\label{eq:ap}
AP_q = \frac{1}{REL_q} \sum_{k=1}^{RET} rel(RET_k)P@k
\end{equation}

\begin{equation}\label{eq:map}
MAP = \frac{1}{Q} \sum_q AP_q
\end{equation}

In information retrieval, average precision is commonly used, since it
allows to consider the order in which documents are returned by a system.
Average precision represents the area under the precision-recall curve
for a given query $q$. Recall for query $q$ is defined as in
Equation~\ref{eq:recall} as the fraction of retrieved, relevant documents
for query $q$ as compared to all relevant documents to query $q$.

Equation~\ref{eq:ap} defines average precision for a given query $q$:
it can be expressed as a finite sum
over the Precision@k values at every possible cut-off $k$, multiplied by an
indicator $rel(RET_k)$, which is 1 if the $k$-th retrieved document is
relevant. This sum is then normalized by the total number of relevant
documents for query $q$ $REL_q$.

Finally, mean average precision $MAP$ represents the mean of all average
precision values over $Q$ queries, as defined in Equation~\ref{eq:map}.

In instance search, mean average precision is not important as we only
care about the first ranked result. Nevertheless, it can be an interesting
metric to allow better comparison between different systems, so we compute
the mean average precision for all results along with the mean Precision@1.

\section{Baselines}
Table %TODO
shows the baselines set for the two evaluation datasets, as described in
Section~\ref{sec:datasets}.

\section{Performance}
\subsection{Training}
\subsection{Evaluation}
% TODO talk about some performance metrics (how much time to train, to evaluate, depending on network. what hardware was used)

% !TeX root = ../full_report.tex

\chapter{Results}
In this section, we present the experimental results obtained in our
research.

\section{Performance}\label{sec:perfresults}
Table~\ref{tab:perftrain} shows the performance of various approaches
when training on the CLICIDE dataset. We assume that pre-trained
networks have 0 cost. This table shows that fine-tuning a network
on classification on the new dataset is quite fast and has a low memory
footprint. This is interesting since fine-tuning already achieves better
results than many standard approaches, as can be seen in
Section~\ref{sec:evalresults}.
We can also see that training our proposed architecture is much
more costly, although the absolute cost is not excessive:
around 48h (TODO) of training time in total for the very deep ResNet-152.

Table~\ref{tab:perftest} shows the performance of these approaches
when evaluating them on a single image. We can see that all approaches
have approximately the same performance, with % TODO.

\begin{table}
\begin{tabular}{|l|c|c|}
\hline & \multicolumn{2}{c|}{\emph{CLICIDE training}}\\
\hline & Time (h) & GPU memory (MB)\\
\hline \emph{AlexNet IN} & 0 & 0\\
\hline \emph{ResNet-152 IN} & 0 & 0\\
\hline \emph{Gordo et al.~\cite{gordo_deep_2016}} & 0 & 0\\
\hline \emph{Gordo et al.~\cite{gordo_deep_2016} (multi-resolution)}
& 0 & 0\\
\hline \emph{AlexNet FT} & 0 & 0\\ % TODO !
\hline \emph{ResNet-152 FT} & 0 & 0\\
\hline \emph{Proposed AlexNet} & 0 & 0\\ % TODO !
\hline \emph{Proposed ResNet-152} & 0 & 0\\
\hline
\end{tabular}
\caption{Performance of different approaches in terms of time in seconds and
maximal GPU memory usage in MB when training on the CLICIDE dataset
\label{tab:perftrain}}
\end{table}

\begin{table}
\begin{tabular}{|l|c|c|}
\hline & \multicolumn{2}{c|}{\emph{CLICIDE test per image}}\\
\hline & Time (h) & GPU memory (MB)\\
\hline \emph{AlexNet IN} & 0 & 0\\ % TODO !
\hline \emph{ResNet-152 IN} & 0 & 0\\
\hline \emph{Gordo et al.~\cite{gordo_deep_2016}} & 0 & 0\\
\hline \emph{Gordo et al.~\cite{gordo_deep_2016} (multi-resolution)}
& 0 & 0\\
\hline \emph{AlexNet FT} & 0 & 0\\ % TODO !
\hline \emph{ResNet-152 FT} & 0 & 0\\
\hline \emph{Proposed AlexNet} & 0 & 0\\ % TODO !
\hline \emph{Proposed ResNet-152} & 0 & 0\\
\hline
\end{tabular}
\caption{Performance of different approaches in terms of time in seconds and
maximal GPU memory usage in MB when testing on a single image
of the CLICIDE dataset\label{tab:perftest}}
\end{table}

Note that the performance reported for testing is not relevant in absolute numbers,
since it was established on a high-end server, usually unavailable when testing
a system. However, the results are relevant in relative terms when comparing
different systems.

% TODO talk about results, what we see etc
% TODO make test on empty decore0 server with one Titan X (possibly extrapolate from 5 epochs or so)

% TODO make test on empty decore0 server with one Titan X
% TODO talk about some performance metrics (how much time to train, to evaluate, depending on network. what hardware was used)

\section{Evaluation results}\label{sec:evalresults}
In this section, we present the results obtained on the presented datasets
using our network as described in Section~\ref{sec:proposed}, as well
as various other approaches.

\begin{table}
\begin{tabular}{|l|c|c|c|c|}
\hline & \multicolumn{2}{c|}{\emph{Mean Precision@1}} &
\multicolumn{2}{c|}{\emph{Mean Average Precision}}\\
\hline & \emph{CLICIDE} & \emph{GaRoFou} & \emph{CLICIDE} & \emph{GaRoFou}\\
\hline \emph{SIFT} & 70.08 & 78.82 & N/A & N/A\\
\hline \emph{AlexNet IN} & 57.58 & 76.63 & 0 & 0\\
\hline \emph{ResNet-152 IN} & 64.24 & 75.54 & 0 & 0\\
\hline \emph{Gordo et al.~\cite{gordo_deep_2016}}
& 90.30 & 95.65 & 0 & 0\\
\hline \emph{Gordo et al.~\cite{gordo_deep_2016} (multi-resolution)}
& 92.73 & 95.65 & 0 & 0\\
\hline \emph{AlexNet FT} & 0 & 0 & 0 & 0\\ % TODO !
\hline \emph{ResNet-152 FT} & 0 & 0 & 0 & 0\\
\hline \emph{Proposed AlexNet} & 0 & 0 & 0 & 0\\ % TODO !
\hline \emph{Proposed ResNet-152} & 0 & 0 & 0 & 0\\
\hline
\end{tabular}
\caption{Evaluation results for the CLICIDE and GaRoFou datasets.
The results are expressed in percentage points of
Mean Precision@1 and mean average precision (only indicative)
\label{tab:results}}
\end{table}

Table~\ref{tab:results} gives an overview of the results obtained. First,
the baselines established in Section~\ref{sec:baselines} are shown
again, with the addition of the relevant results obtained by fine-tuning
a classification network (Section~\ref{sec:finetuning}) and our
proposed network (Section~\ref{sec:proposed}).
In addition to the mean precision@1, we show the mean average
precision obtained by the different approaches.

% TODO  possibly remove this
The comparison of performance between different descriptors should
be based on the mean precision@1. However, the mean average
precision is indicative of how well the model
captures the pattern or characteristics which defines an instance or
sets it apart from other instances. In particular, it is possible to obtain
a high mean precision@1 while achieving a low mean average precision
if the descriptor describes visual similarity well enough. This is because
there is usually at least one reference image that is visually similar
to the query. However, in order to achieve a high mean average precision,
the descriptor needs to capture the specific characteristics of the
instance that set it apart from other instances. For these reasons,
we show the mean average precision obtained by the different
descriptors along with the mean precision@1.

\section{Additional improvements}
In the evaluation results shown above, we consider each approach
as if it provides a strong descriptor for each image, and we only
evaluate the performance of this descriptor.
In information retrieval, many approaches combine descriptors,
or use augmentation techniques in order to improve the results
provided by a single descriptor.

The following sections quickly describe these approaches and
justify whether or not they may improve the results obtained
here.

\subsection{Combination of descriptors}
Many approaches in information retrieval combine multiple
descriptors in order to achieve better results. One of the
major motivations for our research is precisely to avoid
this combination by using an end-to-end approach based
on optimizing a single descriptor globally. On the other hand,
combining descriptors always optimizes the individual descriptors
for a specific task and only later combines them for more
abstract or difficult task. Since we want to avoid choosing
many different descriptors to combine, we do not explore
this approach.

\subsection{Query expansion}
Gordo et al.~\cite{gordo_end--end_2016} use an approach
named query expansion to achieve better results. This
approach essentially performs multiple queries: first, the
descriptor of a query image is used as-is. Then, the descriptor
is combined with the descriptors of the top $k$ retrieved
results, by simply adding the descriptors together. This new
descriptor is used to perform a second query, which gives
the final result.

This technique was introduced by
Chum et al.~\cite{chum_total_2007} and is especially useful
when the descriptor of the query can be spatially matched
against the returned descriptors. This eliminates some of the
returned descriptors from the combined descriptor used
to perform a second query. However, as CNNs produce
global descriptors of images, this spatial matching cannot
be performed.

Furthermore, we do not expect query expansion to provide
any major improvements in our research problem, since
we expect to have very few images returned. This means
the only plausible value of $k$ would be $k=1$. However,
if the best matching descriptor of the first query already
matched, the second query cannot improve the result
and if it does not match, it is unlikely that the second
query would match. Overall, query expansion may
improve mean average precision, but it is unlikely to
improve mean precision@1.

\subsection{Database-side feature augmentation}
Gordo et al.~\cite{gordo_end--end_2016} use another
approach to improve results. This approach is introduced
by Turcot et al.~\cite{turcot_better_2009} and
Arandjelovic et al.~\cite{arandjelovic_three_2012}.

The idea is to combine the descriptors of the reference
images in order to form better database-side descriptors.
Every reference descriptor is simply replaced by a
combination of itself and the $k$ nearest neighbors.
This combination is computed as a weighted sum, weighted
by the rank of the neighbors with respect to $k$ (the
closest neighbor has the highest weight and the $k$-th
neighbor the lowest).
The neighbors can be pre-computed once all descriptors
have been computed for the reference images, by comparing
each image to all others. This can be done efficiently
in our case where descriptors can be compared to other
descriptors using a matrix multiplication.

However, it is difficult to choose the value for $k$, as
it should be low enough to not include many irrelevant
images for the descriptors, but high enough for the
technique to have an impact at all.

\subsection{Instance-averaging}
Since we have the labels of the reference images
available, we propose an approach similar to
database-side feature augmentation in order to
improve the results: for each instance, the descriptors
of all images representing this instance are combined
by averaging the individual descriptors representing
that instance, then normalizing again.

This reduces the number of reference descriptors:
instead of having as many reference descriptors as
reference images, it leaves only as many reference
descriptors as there are instances in the dataset.

\begin{table}
\begin{tabular}{|l|c|c|c|c|}
\hline & \multicolumn{2}{c|}{\emph{Mean Precision@1}} &
\multicolumn{2}{c|}{\emph{Mean Average Precision}}\\
\hline & \emph{CLICIDE} & \emph{GaRoFou} & \emph{CLICIDE} & \emph{GaRoFou}\\
\hline \emph{Gordo et al.~\cite{gordo_deep_2016} (multi-resolution)}
& 92.73 & 95.65 & 0 & 0\\
\hline \emph{Proposed AlexNet} & 0 & 0 & 0 & 0\\ % TODO !
\hline \emph{Proposed AlexNet (instance-averaging)} & 0 & 0 & 0 & 0\\ % TODO !
\hline \emph{Proposed ResNet-152} & 0 & 0 & 0 & 0\\
\hline \emph{Proposed ResNet-152 (instance-averaging)} & 0 & 0 & 0 & 0\\
\hline
\end{tabular}
\caption{Evaluation results for the CLICIDE and GaRoFou datasets,
with and without instance-averaging.
The results are expressed in percentage points of
Mean Precision@1 and mean average precision (only indicative)
\label{tab:instanceavg}}
\end{table}

Table~\ref{tab:instanceavg} shows the results obtained
using the instance-averaging technique. % TODO

% !TeX root = ../full_report.tex

\chapter{Conclusion}
\section{Conclusions from our research}
During our research, we analyzed state-of-the-art approaches in
image classification and image retrieval and their application to
the instance search problem. We found that these approaches struggle
with query images and reference images being at different scales.
Additionally, the region proposals used by the state-of-the-art do not
seem applicable to the problem of instance search.

We have presented a novel approach, which consists of two key steps.
First, we leverage the concept of fully convolutional networks in order
to perform classification training at different scales, without a heavy
computational overhead. Second, we show that the fully convolutional
network can be used to obtain region proposals without the need for
an additional component in the network and training. This is particularly
important, since region proposals are difficult to define manually in
our research problem.

Finally, the proposed network keeps all the benefits of state-of-the-art
approaches: it can be trained end-to-end and it produces an effective
global descriptor, which can be compared using the dot-product.
Additionally, it is modular in the sense that it can be built upon
any type of CNN, pre-trained for classification.
Furthermore, the proposed network is fast to evaluate and the training
time for each dataset is reasonable.

Through multiple experiments on two datasets, we show that the descriptor
obtained using our proposed network outperforms previous state-of-the-art
approaches on the instance search task, while being just as memory-efficient
and fast.

\section{Future work}
\subsection{Few-shots classification}
While we explored different ideas in this report, some have not been
considered yet. For one, a paper by Kaiser et al~\cite{kaiser_learning_2017}
was presented earlier this year, which focuses on few-shots classification
and achieves state-of-the-art results.

As noted before, our research problem shares many similarities with the
few-shots classification problem. Furthermore, this paper presents a
method of combining the triplet loss and cross-entropy loss in order to
organize a memory of so-called rare events. This is similar to our proposed
approach and may be adapted to our research problem. At the very least,
it could provide significant insight to the instance search problem.

\subsection{Conditional similarities}
Another interesting paper presented earlier this year is by
Veit et al~\cite{veit_conditional_2016}. It presents a method of
capturing semantically different notions of similarity. The main idea
is that there are types of similarity (color, category, etc.) and
thus images are embedded into many different subspaces, which capture
these different types of similarity. Finally, they present a method
to jointly learn a global embedding, made up of different sub-embeddings
for the sub-spaces representing different similarities.

This idea could be interesting in the instance search problem, as it
may allow us to learn different types of categories first and then produce
an embedding based on these different categories. For instance, a painting
could be described by its main tone (dark, bright, etc.), its art movement
(impressionism, dadaism, etc.) or simply its size (big, small, squared,
wide, high). Each of these notions may be easily described individually
and a joint learning of these different types of descriptors could provide
significant improvements over learning a single, global descriptor for
each image.

% !TeX root = ../full_report.tex

\appendix \chapter{Appendix}

% TODO possibly add paper here, some more details about implementaiton/where it's hosted/how to run etc

%=========================================================


%=========================================================
\backmatter

\bibliographystyle{plain} % plain-fr si rapport en français
\bibliography{siamese}

%\cleardoublepage % Goes to an odd page
%\pagestyle{empty} % no page number
%~\newpage % goes to a new even page

\end{document}
