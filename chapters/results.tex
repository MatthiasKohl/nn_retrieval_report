% !TeX root = ../full_report.tex

\chapter{Results}
In this section, we present the experimental results obtained in our
research.

\section{Performance}\label{sec:perfresults}
Table~\ref{tab:perftrain} shows the performance of various approaches
when training on the CLICIDE dataset. We assume that pre-trained
networks have 0 cost. This table shows that fine-tuning a network
on classification on the new dataset is quite fast and has a low memory
footprint. This is interesting since fine-tuning already achieves better
results than many standard approaches, as can be seen in
Section~\ref{sec:evalresults}.
We can also see that training our proposed architecture is much
more costly, although the absolute cost is not excessive:
around 48h (TODO) of training time in total for the very deep ResNet-152.

Table~\ref{tab:perftest} shows the performance of these approaches
when evaluating them on a single image. We can see that all approaches
have approximately the same performance, with % TODO.

\begin{table}
\begin{tabular}{|l|c|c|}
\hline & \multicolumn{2}{c|}{\emph{CLICIDE training}}\\
\hline & Time (h) & GPU memory (MB)\\
\hline \emph{AlexNet IN} & 0 & 0\\
\hline \emph{ResNet-152 IN} & 0 & 0\\
\hline \emph{Gordo et al~\cite{gordo_deep_2016}} & 0 & 0\\
\hline \emph{Gordo et al~\cite{gordo_deep_2016} (multi-resolution)}
& 0 & 0\\
\hline \emph{AlexNet FT} & 0 & 0\\ % TODO !
\hline \emph{ResNet-152 FT} & 0 & 0\\
\hline \emph{Proposed AlexNet} & 0 & 0\\ % TODO !
\hline \emph{Proposed ResNet-152} & 0 & 0\\
\hline
\end{tabular}
\caption{Performance of different approaches in terms of time in seconds and
maximal GPU memory usage in MB when training on the CLICIDE dataset
\label{tab:perftrain}}
\end{table}

\begin{table}
\begin{tabular}{|l|c|c|}
\hline & \multicolumn{2}{c|}{\emph{CLICIDE test per image}}\\
\hline & Time (h) & GPU memory (MB)\\
\hline \emph{AlexNet IN} & 0 & 0\\ % TODO !
\hline \emph{ResNet-152 IN} & 0 & 0\\
\hline \emph{Gordo et al~\cite{gordo_deep_2016}} & 0 & 0\\
\hline \emph{Gordo et al~\cite{gordo_deep_2016} (multi-resolution)}
& 0 & 0\\
\hline \emph{AlexNet FT} & 0 & 0\\ % TODO !
\hline \emph{ResNet-152 FT} & 0 & 0\\
\hline \emph{Proposed AlexNet} & 0 & 0\\ % TODO !
\hline \emph{Proposed ResNet-152} & 0 & 0\\
\hline
\end{tabular}
\caption{Performance of different approaches in terms of time in seconds and
maximal GPU memory usage in MB when testing on a single image
of the CLICIDE dataset\label{tab:perftest}}
\end{table}

Note that the performance reported for testing is not relevant in absolute numbers,
since it was established on a high-end server, usually unavailable when testing
a system. However, the results are relevant in relative terms when comparing
different systems.

% TODO talk about results, what we see etc
% TODO make test on empty decore0 server with one Titan X (possibly extrapolate from 5 epochs or so)

% TODO make test on empty decore0 server with one Titan X
% TODO talk about some performance metrics (how much time to train, to evaluate, depending on network. what hardware was used)

\section{Evaluation results}\label{sec:evalresults}
In this section, we present the results obtained on the presented datasets
using our network as described in Section~\ref{sec:proposed}, as well
as various other approaches.

\begin{table}
\begin{tabular}{|l|c|c|c|c|}
\hline & \multicolumn{2}{c|}{\emph{Mean Precision@1}} &
\multicolumn{2}{c|}{\emph{Mean Average Precision}}\\
\hline & \emph{CLICIDE} & \emph{GaRoFou} & \emph{CLICIDE} & \emph{GaRoFou}\\
\hline \emph{SIFT} & 70.08 & 78.82 & N/A & N/A\\
\hline \emph{AlexNet IN} & 57.58 & 76.63 & 0 & 0\\
\hline \emph{ResNet-152 IN} & 64.24 & 75.54 & 0 & 0\\
\hline \emph{Gordo et al~\cite{gordo_deep_2016}}
& 90.30 & 95.65 & 0 & 0\\
\hline \emph{Gordo et al~\cite{gordo_deep_2016} (multi-resolution)}
& 92.73 & 95.65 & 0 & 0\\
\hline \emph{AlexNet FT} & 0 & 0 & 0 & 0\\ % TODO !
\hline \emph{ResNet-152 FT} & 0 & 0 & 0 & 0\\
\hline \emph{Proposed AlexNet} & 0 & 0 & 0 & 0\\ % TODO !
\hline \emph{Proposed ResNet-152} & 0 & 0 & 0 & 0\\
\hline
\end{tabular}
\caption{Evaluation results for the CLICIDE and GaRoFou datasets.
The results are expressed in percentage points of
Mean Precision@1\label{tab:results}}
\end{table}

Table~\ref{tab:results} gives an overview of the results obtained. First,
the baselines established in Section~\ref{sec:baselines} are shown
again, with the addition of the relevant results obtained by fine-tuning
a classification network (Section~\ref{sec:finetuning}) and our
proposed network (Section~\ref{sec:proposed}).
In addition to the mean precision@1, we show the mean average
precision obtained by the different approaches.

% TODO  possibly remove this
The comparison of performance between different descriptors should
be based on the mean precision@1. However, the mean average
precision is indicative of how well the model
captures the pattern or characteristics which defines an instance or
sets it apart from other instances. In particular, it is possible to obtain
a high mean precision@1 while achieving a low mean average precision
if the descriptor describes visual similarity well enough. This is because
there is usually at least one reference image that is visually similar
to the query. However, in order to achieve a high mean average precision,
the descriptor needs to capture the specific characteristics of the
instance that set it apart from other instances. For these reasons,
we show the mean average precision obtained by the different
descriptors along with the mean precision@1.

\section{Additional improvements}
In the evaluation results shown above, we consider each approach
as if it provides a strong descriptor for each image, and we only
evaluate the performance of this descriptor.
In information retrieval, many approaches combine descriptors,
or use augmentation techniques in order to improve the results
provided by a single descriptor.

The following sections quickly describe these approaches and
justify whether or not they may improve the results obtained
here.

\subsection{Combination of descriptors}
Many approaches in information retrieval combine multiple
descriptors in order to achieve better results. One of the
major motivations for our research is precisely to avoid
this combination by using an end-to-end approach based
on optimizing a single descriptor globally. On the other hand,
combining descriptors always optimizes the individual descriptors
for a specific task and only later combines them for more
abstract or difficult task. Since we want to avoid choosing
many different descriptors to combine, we do not explore
this approach.

\subsection{Query expansion}
Gordo et al~\cite{gordo_end--end_2016} use an approach
named query expansion to achieve better results. This
approach essentially performs multiple queries: first, the
descriptor of a query image is used as-is. Then, the descriptor
is combined with the descriptors of the top $k$ retrieved
results, by simply adding the descriptors together. This new
descriptor is used to perform a second query, which gives
the final result.

This technique was introduced by
Chum et al~\cite{chum_total_2007} and is especially useful
when the descriptor of the query can be spatially matched
against the returned descriptors. This eliminates some of the
returned descriptors from the combined descriptor used
to perform a second query. However, as CNNs produce
global descriptors of images, this spatial matching cannot
be performed.

Furthermore, we do not expect query expansion to provide
any major improvements in our research problem, since
we expect to have very few images returned. This means
the only plausible value of $k$ would be $k=1$. However,
if the best matching descriptor of the first query already
matched, the second query cannot improve the result
and if it does not match, it is unlikely that the second
query would match. Overall, query expansion may
improve mean average precision, but it is unlikely to
improve mean precision@1.

\subsection{Database-side feature augmentation}
Gordo et al~\cite{gordo_end--end_2016} use another
approach to improve results. This approach is introduced
by Turcot et al~\cite{turcot_better_2009} and
Arandjelovic et al~\cite{arandjelovic_three_2012}.

The idea is to combine the descriptors of the reference
images in order to form better database-side descriptors.
Every reference descriptor is simply replaced by a
combination of itself and the $k$ nearest neighbors.
This combination is computed as a weighted sum, weighted
by the rank of the neighbors with respect to $k$ (the
closest neighbor has the highest weight and the $k$-th
neighbor the lowest).
The neighbors can be pre-computed once all descriptors
have been computed for the reference images, by comparing
each image to all others. This can be done efficiently
in our case where descriptors can be compared to other
descriptors using a matrix multiplication.

However, it is difficult to choose the value for $k$, as
it should be low enough to not include many irrelevant
images for the descriptors, but high enough for the
technique to have an impact at all.

\subsection{Instance-averaging}
Since we have the labels of the reference images
available, we propose an approach similar to
database-side feature augmentation in order to
improve the results: for each instance, the descriptors
of all images representing this instance are combined
by averaging the individual descriptors representing
that instance, then normalizing again.

This reduces the number of reference descriptors:
instead of having as many reference descriptors as
reference images, it leaves only as many reference
descriptors as there are instances in the dataset.

Table~\ref{TODO} shows the results obtained
using this additional technique. % TODO
