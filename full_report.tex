\documentclass[fleqn]{article}

%% Language and font encodings
% \renewcommand{\familydefault}{\sfdefault} % use sans serif per default
% \usepackage{helvet} % use helvetica per default

\usepackage[english]{babel}
\usepackage[utf8]{inputenc}
\usepackage[T1]{fontenc}

%% Sets page size and margins
% \usepackage[a4paper,top=1.5cm,bottom=1.5cm,left=2cm,right=2cm,marginparwidth=1.75cm]{geometry}

%% Useful packages
\usepackage{amsmath,amssymb} % for advanced math formulas
\setlength{\mathindent}{0pt}
\usepackage{hyperref}
\usepackage{graphicx}

\title{Deep learning with Siamese networks for instance search or identification}
\author{Matthias Kohl}

\begin{document}
\maketitle

\begin{abstract}
TODO
\end{abstract}

\section{Introduction}
\subsection{Motivation}
The research presented here is motivated by the GUIMUTEIC project,
a collaborative project between industry and LIG. The
aim is to develop a smart audio-guide for touristic or cultural sites.
TODO cite project/LIG/etc

In practice, the final product should offer an augmented reality
interface to the user, with information about the object or objects
the user is looking at.

One part of the development consists in finding ways of identifying
objects the user is looking at. There are multiple possibilities,
for example based on the geo-localization of the user and other sensors.
In our research, we focus on the recognition of objects based
only on visual clues.

\subsection{Research problem}
More specifically, we are interested in the following problem:
We are given a collection with reference images for each object,
or instance, to be recognized. The task is to develop a system that,
given an image of one of the instances, can decide which instance
the image represents. We assume that, on average, less than ten images
are available for each instance.
We will refer to this problem as instance retrieval in the following.

There are a few problems similar to instance retrieval. For one, there is
the image classification problem: we are given a collection of images
where each image is assigned a class, like dog or fridge.
The task is to develop a system that, given an image, decides which
class the image represents.
This problem is of course similar if we simply consider each instance
to be a separate class. However, in classification, we usually
assume to have many images per class: hundreds or even thousands.
This means our problem is closer to few-shots classification, where
only few images are available for each class.

Another similar problem is image retrieval. In image retrieval,
we are given a collection of reference images and a query image.
We aim to rank the reference images by similarity to the query image.
Usually, in image retrieval, reference images are not labeled or loosely
labeled and many images are irrelevant to the query image. The challenge
is to rank the most similar images on top.
Instance retrieval is related to image retrieval in that we aim to
develop a notion of similarity between images. However, in instance
retrieval, we do not care about the rank of the returned images, since
we only consider the highest ranked image, which should represent the
instance to retrieve.

% TODO possibly elaborate here

\subsection{Challenges}
For all of the problems described above, deep learning approaches
based on convolutional neural networks (CNNs) have recently obtained
the state-of-the-art results. One of the drawbacks of CNNs is that
they require large amounts of data to be trained.

In our research problem in particular, we do not
have large enough amounts of data available to fully train a CNN.
However, we aim to develop a system that can learn a descriptor
specific to the reference images,
as the system only needs to recognize and differentiate between
instances of that particular dataset. So the system should be tuned to
the dataset. Since we usually have less than ten images per instance,
this is an important challenge to overcome.

Another challenge comes from the nature of the datasets: we were not
able to find similar datasets in the literature, with only a few,
clean images, for each instance and many instances.
Common datasets in image retrieval like Oxford5k [TODO], Paris6k [TODO]
or Holidays [TODO] contain many images per instance, but a lot of noise,
as they are used to evaluate whether a system can robustly retrieve the
correct images out of a set of random images.
This means that the approaches used for image retrieval datasets may
not work as well for instance retrieval.

\subsection{Contributions}
We made the following three major contributions:
\begin{enumerate}
    \item We analyze existing approaches to image retrieval, classification
    and determine their shortcomings in our problem setting.
    \item Based on these shortcomings, we propose a novel approach
    to improve results in our problem setting.
    \item We evaluate the novel approach and compare with other approaches.
\end{enumerate}

\section{State of the art}
Previously, most systems were based on a bag-of-words approach
in order to match images~\cite{philbin_object_2007}.
Combined with better techniques of matching the bag-of-words descriptors,
this approach was previously the state-of-the-art~\cite{mikulik_learning_2013}.

Recently, the state-of-the-art has drastically improved due to the
addition of learned features, based on convolutional neural networks
(CNNs).
This new approach greatly improved
mean average precision scores~\cite{gordo_deep_2016} of the system
on the same datasets.

The general trend is to move from an approach based on a combination of
engineered features (bag-of-words combined with
support vector machines or SVMs) to a more end-to-end learning of
the matching of two images.

For this, the Siamese architecture is used, where the convolutional
features of two or more CNNs with shared weights are combined and a
multi-way loss is optimized, which discriminates between the features
of two or more images.
This concept was first introduced by
Chopra et al~\cite{chopra_learning_2005} to learn a dissimilarity metric
between two images.

For a learning-to-rank setting, this idea was extended to a triplet loss
by Weinberger et al~\cite{weinberger_distance_2006}.
This loss is better suited when learning to rank, rather than instance
search, but it also allows for a more stable convergence, which is
beneficial in both cases.

Using this triplet loss has achieved state-of-the-art results in both
face recognition~\cite{schroff_facenet:_2015} and
image retrieval~\cite{gordo_deep_2016}. This is mostly due to the usage
of far deeper CNNs, moving from architectures such as
AlexNet~\cite{krizhevsky_imagenet_2012} to VGG~\cite{simonyan_very_2014}
and finally Inception~\cite{szegedy_inception-v4_2016} and
ResNet~\cite{he_deep_2015}.

The higher depth of networks like ResNet and Inception as compared to
AlexNet allows for higher regularization and thus less over-fitting
to a specific dataset for these architectures. Batch normalization
increases this effect even more. This is a desirable trait for image
retrieval, as the variability within each instance is too small to
learn classification directly. So the better we can generalize
features learned from a bigger dataset to the reference dataset,
the better performance we can expect.

\section{Evaluation}\label{sec:eval}
We are using the following datasets in our experiments:
\begin{enumerate}
    \item CLICIDE: dataset of photographs taken in an art museum,
    consisting exclusively of paintings. The dataset is characteristic
    because the different images for each instance consist of one
    global view of the instance and multiple sub-parts.

    Number of reference images: 3245 (the full dataset contains
    3452 images with 207 images of walls and no meaningful instance)
    Number of test images: 165 (the full dataset contains 177
    test images, out of which 165 share their instance with
    at least one image of the reference set)
    Number of instances: 464
    \item GaRoFou\_I: dataset of photographs of an art museum,
    consisting of display cabinets, which contain sculptures,
    rocks and various types of objects.

    Number of reference images: 1068
    Number of test images: 184 (all are used for evaluation)
    Number of instances: 311
    % \item GaRoFou_V: dataset of frames from a video taken by visitors
    % of the same museum as above.

    % Number of reference images: 2779
    % Number of test images: 2132
    % Number of instances: 215
    % \item Oxford5k: dataset of photographs of buildings taken in
    % the city of Oxford, UK.

    % Number of reference images: 5000 TODO
    % Number of test images: 55 TODO
    % Number of instances: 17 TODO
\end{enumerate}

\subsection{Evaluation metrics}
\subsubsection{Metrics definitions}
\paragraph{$Precision@k$}
For a test image and $m$ reference images and a ranking of the
reference images $I_1, \dots, I_m$, we define the number of relevant
images at $k$ with $k \leq m$:
$N^{rel}_k$ is the number of images in the sub-ranking
$I_1, \dots, I_k$ from the same instance than the test image.

We then define Precision at $k$ as:

\begin{equation}
Precision@k = \frac{N^{rel}_k}{k}
\end{equation}

\paragraph{MAP}
As above, for a test image, $m$ reference images and a ranking
of the reference images, we define the average precision:

\begin{equation}
AP = \frac{1}{N^{rel}_m} \sum_{k=1}^m Precision@k
\end{equation}

The MAP score is defined as the mean of the AP score over all test images.

\subsubsection{Used metrics}
As of now, we use the mean $Precision@1$ to evaluate the system.
It is the mean of the $Precision@1$ for all test images.
This is what we are most interested in: in the context of a
search system for retrieving art in a museum, we are only
interested in one result, as the user should not make a choice
out of several results.

To be comparable with other papers in the field, we will implement
the MAP score as well.

\bibliographystyle{ieeetr}
\bibliography{siamese}

% \newpage
% \appendix
% \section{Implementation details and specifics}
% \subsection{General issues}

\end{document}
